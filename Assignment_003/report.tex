%
% @author   Shmish  "shmish90@gmail.com"
% @legal    MIT     "(c) Christopher Schmitt"
%


\documentclass{article}


%
% Document Imports
%

\usepackage{fancyhdr}
\usepackage{extramarks}
\usepackage{amsmath}
\usepackage{amssymb}
\usepackage{amsthm}
\usepackage{amsfonts}
\usepackage{algpseudocode}
\usepackage[table,xcdraw]{xcolor}
\usepackage{color}
\usepackage{tikz}
\usepackage{forest}
\usepackage{listings}


\definecolor{dkgreen}{rgb}{0,0.6,0}
\definecolor{gray}{rgb}{0.5,0.5,0.5}
\definecolor{mauve}{rgb}{0.58,0,0.82}


\lstset{
  frame=tb,
  language=Java,
  aboveskip=3mm,
  belowskip=3mm,
  showstringspaces=false,
  columns=flexible,
  basicstyle={\tiny\ttfamily},
  numbers=none,
  numberstyle=\tiny\color{gray},
  keywordstyle=\color{blue},
  commentstyle=\color{dkgreen},
  stringstyle=\color{mauve},
  breaklines=true,
  breakatwhitespace=true,
  tabsize=3
}


%
% Document Configuation
%

\newcommand{\hwAuthor}{Christopher Schmitt}
\newcommand{\hwSubject}{CS 253}
\newcommand{\hwSection}{Section 02}
\newcommand{\hwSemester}{Fall 2019}
\newcommand{\hwAssignment}{Assignment 3}


%
% Document Enviornments
%

\setlength{\headheight}{65pt}
\pagestyle{fancy}
\lhead{\hwAuthor}
\rhead{
  \hwSubject \\
  \hwSection \\
  \hwSemester \\
  \hwAssignment
}

\newenvironment{problem}[1]{
  \nobreak\section*{Problem #1}
}{}


%
% Document Start
%

\begin{document}
  \begin{center}
    \begin{forest}
      [H
        [F
          [E
            [C
              [$\emptyset$]
              [B
                [$\emptyset$]
                [A]
              ]
            ]
            [D]
          ]
          [$\emptyset$]
        ]
        [G
          [I]
          [J
            [K]
            [$\emptyset$]
          ]
        ]
      ]
    \end{forest}
  \end{center}
  By observing the postorder and inorder sequences, the origanl binary tree can
  be reconstructed.  H will be the root node, as it is the last element od the
  postorder set.  Everything to the right of H in the inorder set must be to
  the right of H in the tree.  Likewise, everything on the left of H in the
  inorder set must be to the left of H in the tree.  The inorder set can now be
  split into two sets, and the entire process can recur on the two sub-sets.
\end{document}